\title{Physics 16 Final Project}
\newcommand{\f}{\frac} 
\newcommand{\om}{\omega}
\newcommand{\pdif}[2]{\f{\partial #1}{\partial #2}}
\newcommand{\dif}[2]{\f{d#1}{d#2}}
\newcommand{\eqdef}{\equiv}
\documentclass[12pt,letterpaper]{article}

\usepackage{graphicx}
\usepackage{caption}
\usepackage{subcaption}
\usepackage{amsmath} 
\usepackage{amssymb}
\usepackage{ulem}
\usepackage{tikz}
\usepackage{multicol}
\usepackage[left=1in,top=1in,right=1in,bottom=1in,nohead]{geometry}
\usetikzlibrary{decorations.markings}
\usetikzlibrary{decorations.pathreplacing}

\usepackage{amsthm} 
\usepackage{wrapfig}
\usepackage{enumitem}
%\usepackage{enumerate}
\newtheorem{mydef}{Definition}
\newtheorem{example}{Example}
\newtheorem{thrm}{Theorem}
\newtheorem{lemma}{Lemma}
\newtheorem{cor}{Corollary}
\newtheorem{notation}{Notation}
\newtheorem{rem}{Remarks}
\newcommand{\biu}[1]{\underline{\textbf{\textit{#1}}}}
\newcommand{\so}{\Rightarrow}
\newcommand{\Lagr}{\mathcal{L}}
\usepackage[ampersand]{easylist}

\let\oldemptyset\emptyset
\let\emptyset\varnothing

\author{Keno Fischer}

\newcommand{\homework}{\biu{Homework}}
\newcommand{\Mor}{\text{Mor}}
\newcommand{\N}{\mathbb{N}}
\newcommand{\Q}{\mathbb{Q}}
\newcommand{\Z}{\mathbb{Z}}
\newcommand{\R}{\mathbb{R}}
\newcommand{\C}{\mathbb{C}}
\newcommand{\pabs}[1]{\left|\left| #1 \right|\right|_p}
\newcommand{\set}[1]{\left\{#1 \right\}}
\newcommand{\paren}[1]{\left(#1 \right)}
\newcommand{\parens}{\paren}
\newcommand{\tot}[0]{\text{tot}}
%\newcommand{\pabs}[1]{#1}
\begin{document}
\tikzstyle{lattice}=[shape=circle,draw,fill,text=white]
\tikzset{
  % style to apply some styles to each segment of a path
  on each segment/.style={
    decorate,
    decoration={
      show path construction,
      moveto code={},
      lineto code={
        \path [#1]
        (\tikzinputsegmentfirst) -- (\tikzinputsegmentlast);
      },
      curveto code={
        \path [#1] (\tikzinputsegmentfirst)
        .. controls
        (\tikzinputsegmentsupporta) and (\tikzinputsegmentsupportb)
        ..
        (\tikzinputsegmentlast);
      },
      closepath code={
        \path [#1]
        (\tikzinputsegmentfirst) -- (\tikzinputsegmentlast);
      },
    },
  },
  % style to add an arrow in the middle of a path
  end arrow/.style={postaction={decorate,decoration={
        markings,
        mark=at position 1 with {\arrow[#1]{stealth}}
      }}},
}

\paragraph{Question 1}
\subparagraph*{Part 1-everyone-1}
We will assume that in the cylindrical the speed of light is still equal to $c$ in all initial reference frames. However, as we will see, it is not necessarily true anymore that the any two initial reference are identical.  \par 
In this universe, sending out a light pulse simultaneously will result in it coming back from both the $+x$ and the $-x$ direction, after two different time intervals $T_+$ and $T_-$. It is easy to see conceptually how such an asymmetry can arise in a cylindrical universe. Suppose our base ship is moving at some velocity $u$ in the $+x$ direction and sends out a light pulse at some time $t_0$. Since $c$ is finite, the ship will continue to travel in the $+x$ direction while the light travels all the way around. However, because of that, the light that was sent in the $-x$ direction will meet the ship earlier than the light sent into the $+x$ direction (since it had to travel less and both light pulses move at the same speed). \par 
A perhaps easier and more intuitive way to think about this is to recall the use of simultaneity of events in inertial reference frames: Given some local spacetime event in our initial reference frame, all those events are simultaneous which would be reached by a light pulse emitted directly at the original spacetime event. Thus at any given instant, our ship is simultaneous with a copy of itself, whose internal clock has already advanced by $T_+$ and is located at a distance of $T_+c$ in the $+x$ direction and another copy of itself whose internal clock has advanced by $T_-$ located a distance of $T_-$ in the $-x$ direction. \par 
The analysis of the twin paradox becomes now rather routine: Suppose we are sending out a probe in the $+x$ direction. Then, to reach the ship in the positive $x$ direction, the probe has to move a distance of $T_+c$. Due to length contraction, according to a clock on the probe, the journey takes a total time of \[t=\frac{T_+c}{v\gamma} = T_+\sqrt{c^2/v^2-1}\], on the other hand, seen from the copy of the ship at $x=T_+x$, the probe is emitted by the ship in the middle, when the internal clock of the "copy" has already advanced by $T_+$. Thus the time as seen in the frame of the ship is given by
\[ t_s = T_+(1+\frac{c}{v}) \]
\subparagraph*{Remark}
In this universe, the second postulate of special relativity, no longer holds in its full generality as the values of $T_+$ and $T_-$ will be different for every reference frame (see the argument for why such a discrepancy may exist in the first place for a reasoning why this should be the case). In particular, there exists one very special from in which $T_+=T_-$. This frame could easily be called the total rest frame of the universe, since the asymmetry in $T_+$ and $T_-$ can be easily explained by objects moving relative to this reference frame.
\subparagraph*{Part 1a)}
\paragraph{Question 2}
\subparagraph{2-Everyone}
First, let us determine $x_1$ and $x_2$ in terms of $x$; manipulating the given equations,
$$x_1(t)(\f{m_2}{m_1+m_2})-x_2(t)(\f{m_2}{m_1+m_2})=\f{m_2}{m_1+m_2} C\sin(\om_d t)$$
Adding this to the second equation gives us
$$x_1(t)=x(t)+\f{m_2}{m_1+m_2} C\sin(\om_d t)$$
Similarly, we have the other relation
$$x_2(t)=x(t)-\f{m_1}{m_1+m_2}C\sin(\om_d t)$$
Now, we calculate the Lagrangian in terms of $x_1$ and $x_2$ and then convert
$$L=T-V=\f 1 2 m_1 \dot{x_1}^2+\f 1 2 m_2 \dot{x_2}^2-(\f 1 2 K_1 x_1^2+\f 1 2 K_2 x_2^2)$$
$$L=\f 1 2 m_1(\dot x(t)+\f{m_2}{m_1+m_2} C\om_d \cos(\om_d t))^2+\f 1 2 m_2(\dot x(t) -\f{m_1}{m_1+m_2} C\om_d \cos(\om_d t))^2$$
$$-(\f 1 2 K_1 (x(t)+\f{m_2}{m_1+m_2} C\sin(\om_d t))^2+\f 1 2 K_2 ( x(t)-\f{m_1}{m_1+m_2} C\sin(\om_d t))^2)$$
Next, we set up the Euler-Lagrange equation.
$$\pdif L x=-K_1 (x(t)+\f{m_2}{m_1+m_2}C\sin(\om_d t))-K_2 (x(t)-\f{m_1}{m_1+m_2} C\sin(\om_d t))$$
$$=-(K_1+K_2) x(t)-\f{K_1 m_2-K_2 m_1}{m_1+m_2}C \sin(\om_d t)$$
$$\pdif L {\dot x} = m_1 (\dot x(t)+\f{m_2}{m_1+m_2} C \om_d \cos(\om_d t))+m_2  (\dot x(t)-\f{m_1}{m_1+m_2} C \om_d \cos(\om_d t))$$
$$= (m_1+m_2) \dot x(t)$$
Our Euler-Lagrange equation is then
$$\ddot x(t) (m_1+m_2)+ x(t) (K_1+K_2)=\f{K_2 m_1-K_1 m_2}{m_1+m_2} C \sin(\om_d t)$$
The corresponding homogeneous equation has the general solution
$$x(t)= a \sin(\om_d t)+b\cos(\om_d t)$$
for $\om_d$ as defined in the prompt.  Therefore, to solve for the general solution to this differential equation, we only need to find a single solution.  Suppose we have an ansatz of
$$x(t)=A t\cos(\om_d t)$$
Then, 
$$x'(t)= A\cos(\om_d t)-A\om_d t\sin(\om_d t)$$
$$x''(t)=-2A\om_d \sin(\om_d t)-A\om_d^2 t \cos(\om_d t)$$
Plugging this into the equation from earlier, we get
$$(m_1+m_2)(-2A\om_d \sin(\om_d t)-A\om_d^2 t \cos(\om_d t))$$
$$+(K_1+K_2)( A t \cos(\om_d t))=\f{K_2 m_1-K_1 m_2}{m_1+m_2} C \sin(\om_d t)$$
We know that $\om_d^2 = \f{K_1+K_2}{m_1+m_2}$, so this equation is equivalent to
$$(m_1+m_2)(-2A\om_d \sin(\om_d t))=\f{K_2 m_1-K_1 m_2}{m_1+m_2} C \sin(\om_d t)$$
which is itself equivalent to the condition that
$$A=C\f{K_1 m_2-K_2 m_1}{2\om_d(m_1+m_2)^2}=C\f{K_1 m_2-K_2 m_1}{2(K_1+K_2)^{\f 1 2}(m_1+m_2)^{\f 3 2}}$$
Hence, our general solution to the differential equation is
$$x(t)=A t\cos(\om_d t)+a \cos(\om_d t)+b\sin(\om_d t)$$
where $A$ is the $A$ we just calculated, and $a$ and $b$ are not fixed.  Plugging in our first initial condition ($x(0)=0$) we get that $a=0$.  Therefore, $x(t)=At\cos(\om_d t)+b\sin(\om_d t)\rightarrow \dot x(t)=-At \om_d \sin(\om_d t)+ A\cos(\om_d t)+b\om_d\cos(\om_d t)$.  Plugging in our initial condition $\dot x(0)=0$ then tells us that
$$b=\f{A}{\om_d}=C\f{K_1 m_2-K_2 m_1}{2(K_1+K_2)(m_1+m_2)}$$
Now, we have completely described $x(t)$.  Our earlier equations relating $x_1$ and $x_2$ to $x$ still hold, so given our description of $x(t)$, we find that (for our calculated values of $A$ and $b$)
$$x_1(t)=At \cos(\om_d t)+b\sin(\om_d t)+\f{m_2}{m_1+m_2} C\sin(\om_d t)$$
$$x_2(t)=At\cos(\om_d t)+b\sin(\om_d t)-\f{m_1}{m_1+m_2}C\sin(\om_d t)$$
\subparagraph{2-a}  First, we relate $x_1$ and $x_2$:
$$x_2(t)= x_1(t)-C\sin(\om_d) t$$
Then, we compute the Lagrangian:
$$L=T-V= \f 1 2 (m_1 \dot x_1^2 + m_2 (\dot x_1 - C\om_d \cos(\om_d t))^2)-\f 1 2 (K_1 x_1^2+K_2 (x_1-C \sin(\om_d t)^2))$$
We then compute
$$\pdif L {\dot x_1}= m_1 \dot x_1 + m_2 (\dot x_1 - C \om_d \cos(\om_d t))$$
$$\pdif L {x_1}=-(K_1 x_1+K_2(x_1-C\sin (\om_d t)))$$
The Euler-Lagrange equation is then 
$$  (m_1+m_2)\ddot x_1 + C m_2 \om_d^2 \cos(\om_d t)=-((K_1+K_2) x_1 - C K_2 \sin(\om_d t))$$
\subparagraph{2-b}  The relation between $x_1$ and $x_2$ we will be using is
$$x_1(t)=x_2(t) + C\sin(\om_d t)$$
Therefore, the Lagrangian for the system is 
$$L=T-V= \f 1 2 (m_2 \dot x_2^2 + m_1 (\dot x_2 + C\om_d \cos(\om_d t))^2)-\f 1 2 (K_2 x_2^2+K_1 (x_2+C \sin(\om_d t)^2))$$
We then compute
$$\pdif L {\dot x_2}= m_2 \dot x_2 + m_1 (\dot x_2+ C \om_d \cos(\om_d t))$$
$$\pdif L {x_2}=-(K_2 x_2+K_1(x_2+C\sin (\om_d t)))$$
The Euler-Lagrange equation is then
$$ (m_1+m_2) \ddot x_2-m_1 C \om_d^2 \sin(\om_d t)=-((K_1+K_2) x_2+C K_1 \sin(\om_d t))$$

\paragraph{Question 3}
\subparagraph{Part 1}
We will assume that Shaun starts in a horizontal position (i.e. the board is vertical) and that his center of mass is located at a height of $R$ above the bottom of the circle. Such an assumption is reasonable, because any other initial configuration would only add a constant term of additional energy which would not qualitatively affect the analysis below.

Thus by the initial conditions we have that Shaun has initial energies of
\[ KE_0 = 0 \]
\[ PE_0 = mgR \]
Now, note that we cannot just use the conservation of energy directly as there is work being done in the instance that Shaun rises and we do not now (yet) what the magnitude of this work is. To find said work, consider the angular momentum about the center of the circle (which must be equal before and after the instantaneous rise since the force is along the axis between the origin of the circle and Shaun's center of mass):
\[ \vec{L_A} = \vec r \times \vec p = rmv \hat{z} \]
where $\hat{z}$ is pointing into the plane. \par
Now, we may easily find the velocity at the instantaneous moment before the rise $v_A$ by analyzing the energy (and thereby using the fact that energy is conserved in the absence of friction.
\[ v_A=\sqrt{2\frac{KE_A}{m}} = \sqrt{2\frac{PE_0-PE_A}{m}} = \sqrt{2g(R-(l-h))}  \]
We now that angular momentum is conserved in this instant and thus the speed afterwards $v_B$ is given by
\[ v_B = \frac{r_A}{r_B} v_A = \frac{l-h}{R-(l-h)} v_A = \left(1+\frac{h}{R-l}\right) v_a \]
So we have
\[ KE_B = KE_A\left(1+\frac{h}{R-l}\right)^2 \]
\[ PE_B = mgl \]
Now, we may find the distance of Shaun ascend over the half pipe by comparing this new total energy with the potential energy at his highest ascend, so we get, 
\[ d=\frac{KE_B+PE_B}{mg}-R = \left(R-l+h\right)\left(1+\frac{h}{R-l}\right)^2+l-R\]
To first order in small dimensionless quantities, 
\[ \left(R-l+h\right)\left(1+\frac{h}{R-l}\right)^2 = \left(R-l+h\right) \left(1+\frac{2h}{R-l}+\frac{h^2}{(R-l)^2}\right) \approx (R-l+h)+(2h) \]
so we get
\[ d = l-R + R - l + 3h =3h \]
\subparagraph{Part 2}


\subparagraph{Part 3a)}
Since there is no friction, we know that the only forces acting on Shaun are the normal force and gravity. Now, consider the situation in which Shaun's center of mass is located some distance $z$ below the top of the half pipe. Then we know that
\[ z=R\cos(\theta) \]
\[ F_g = F_n\cos\theta \] 
so the magnitude of $F_n$ is 
\[ F_n = \frac{F_g}{\cos\theta} = \frac{RF_g}{z} \]



\paragraph{Question 3 - 1}
We will start by considering Shaun being positioned exactly horizontally (which we may assume since $w \ll R$ in the upper right corner of the halfpipe.

\paragraph{Question 4.}

\subparagraph{Part 1.}  Moment-of-inertia tensors about the same point (in this case, the shared center of mass) simply add, so the total moment-of-inertia tensor is $$\overleftrightarrow{I}
= \overleftrightarrow{I^o} + \overleftrightarrow{I^i} =
\begin{bmatrix}
I_\bot^i + I_\bot^o & 0 & 0 \\
0 & I_\bot^i + I_\bot^o & 0 \\
0 & 0 & I_3^i + I_3^o
\end{bmatrix}$$

\subparagraph{Part 2.}

Though $\vec{L} \eqdef \vec{L}^i + \vec{L}^o$ stays constant in a fixed space frame, the angular momenta $\vec{L}^i$ and $\vec{L}^o$ of the parts of the station do not necessarily stay constant, because each part of the station exerts a torque on the other part.  Fix a coordinate frame in space such that $\vec{L}$ points parallel to $\hat{z}$, $\hat{y}$ points parallel to $\hat{e}_2$, and the positive $xz$-quadrant contains $\vec{\omega}$. When the two parts separate, $\vec{L}^i$ and $\vec{L}^o$ thus define new axes of precession for the parts of the spaceship.  The separated parts begin to precess about these new axes, instantaneously changing the angles and possibly directions of precession.  The inner part precesses with angular velocity $\tilde{\omega}^i = L^i/I_\bot^i$ and the outer part precesses with angular velocity $\tilde{\omega}^o = L^o/I_{\bot}^o$, where $L^i$ without superscript is the magnitude of $\vec{L^i}$, and likewise for $L^o$.

\subparagraph{Part 3.}

The axes of symmetry $\hat{e}_3^i$ and $\hat{e}_3^o$ precess in circles with one point of exterior tangency at the breakup point and no other points of intersection.  The two parts of the space station will therefore align again after each component has completed an integer number of revolutions, which requires the quantity $\tilde{\omega}^i / \tilde{\omega}^o$ to be rational.

For the remainder of this problem, let $\omega_3$ and $\omega_\bot$ be the components of $\vec{\omega}$ along $\hat{e}_3$ and $\hat{e}_1$, respectively.  We can compute the magnitudes of $\vec{L}^i$ and $\vec{L}^o$ just as well in the body frame as in the fixed frame.  In the body frame, we have $$\vec{L}^i = \overleftrightarrow{I^i}\vec{\omega} = \omega_\bot I^i_\bot \hat{e}_1 + \omega_3 I_3 \hat{e}_3$$ and thus $$L^i = \sqrt{\omega_\bot^2 (I_\bot^i)^2 + \omega_3^2 (I_3^i)^2}.$$  The formula for $L^o$ repeats this analysis, swapping superscript $o$ for superscript $i$.  Thus, the axes of symmetry of the two parts of the space station become parallel again only if $$\frac{\tilde{\omega}^i}{\tilde{\omega}^o} = \frac{L^o I^i_\bot}{L^i I^o_\bot} = \frac{I^i_\bot \sqrt{\omega_\bot^2 (I_\bot^o)^2 + \omega_3^2 (I_3^o)^2}}{I^o_\bot \sqrt{\omega_\bot^2 (I_\bot^i)^2 + \omega_3^2 (I_3^i)^2}}$$ is rational.  If we let $\theta$ be the angle formed by $\vec{\omega}$ in the $\hat{e}_1 \hat{e}_3$ plane (such that $\omega_3 = \omega \sin \theta$ and $\omega_\bot = \omega \cos \theta$), then we have this perhaps nicer-looking version of the same quantity: $$\frac{I^i_\bot \sqrt{(I^o_\bot )^2 \cos^2 \theta + (I^o_3)^2 \sin^2 \theta}}{I^o_\bot \sqrt{(I^i_\bot)^2 \cos^2 \theta + (I^i_3)^2 \sin^2 \theta}}.$$

\subparagraph{Part 4.}  As the horizontal rods are massless and rigid, any torque exerted on the horizontal rods will be transferred to the inner control center.  To compute the torque, therefore, it suffices to compute $d\vec{L^i}/dt$ for $t = 0$ in the fixed space frame.  We apply the equation $d\vec{L}^i/dt = \partial \vec{L}^i/\partial t + \omega \times \vec{L}^i$.

Consider the first term first.  We have $\vec{\omega} = \omega_\bot \cos (\Omega t + \phi) \hat{e}_1 + \omega_\bot \sin (\Omega t + \phi) \hat{e}_2 + \omega_3 \hat{e}_3$, where $\Omega = \omega_3 (I^i_3 - I^i_\bot)/I^i_\bot$.  Our choice of coordinate system gives us $\phi = 0$, so $\dot{\vec{\omega}}(0) = \omega_\bot \Omega \hat{e}_2$.  As $\vec{L}^i = \overleftrightarrow{I^i} \vec{\omega}$ in the body frame, evaluating $\partial \vec{L}^i/\partial t$ at $t = 0$ gives us $\overleftrightarrow{I^i} \dot{\vec{\omega}}(0) = I^i_\bot \Omega \omega_\bot \hat{e}_2 = (I^i_3 - I^i_\bot) \omega_\bot \omega_3 \hat{e}_2$.

Now consider the second term.  We can take the cross product in the body frame, in which $\vec{L^i} = \overleftrightarrow{I^i} \vec{\omega}$ is still valid.  At $t = 0$, $\vec{\omega} = \omega_\bot \hat{e}_1 + \omega_3 \hat{e}_3$, so $\vec{L^i} = I^i_\bot \omega_\bot \hat{e}^1 + I^i_3 \omega_3 \hat{e}_3$ and $\omega \times \vec{L^i} = - (I^i_3 + I^i_\bot) \omega_\bot \omega_3 \hat{e}_2$.  This gives us our final answer for the torque: $$\boxed{\tau = - 2 I^i_\bot \omega_\bot \omega_e \hat{e}_2.}$$

\end{document}
