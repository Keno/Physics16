\title{Physics 16 Final Project}
\newcommand{\f}{\frac} 
\newcommand{\om}{\omega}
\newcommand{\pdif}[2]{\f{\partial #1}{\partial #2}}
\newcommand{\dif}[2]{\f{d#1}{d#2}}
\newcommand{\eqdef}{\equiv}
\documentclass[12pt,letterpaper]{article}

\usepackage{graphicx}
\usepackage{caption}
\usepackage{subcaption}
\usepackage{amsmath} 
\usepackage{amssymb}
\usepackage{ulem}
\usepackage{tikz}
\usepackage{multicol}
\usepackage{verbatim}
\usepackage[left=1in,top=1in,right=1in,bottom=1in,nohead]{geometry}
\usetikzlibrary{decorations.markings}
\usetikzlibrary{decorations.pathreplacing}

\usepackage{amsthm} 
\usepackage{wrapfig}
\usepackage{enumitem}
%\usepackage{enumerate}
\newtheorem{mydef}{Definition}
\newtheorem{example}{Example}
\newtheorem{thrm}{Theorem}
\newtheorem{lemma}{Lemma}
\newtheorem{cor}{Corollary}
\newtheorem{notation}{Notation}
\newtheorem{rem}{Remarks}
\newcommand{\biu}[1]{\underline{\textbf{\textit{#1}}}}
\newcommand{\so}{\Rightarrow}
\newcommand{\Lagr}{\mathcal{L}}
\usepackage[ampersand]{easylist}

\let\oldemptyset\emptyset
\let\emptyset\varnothing

\author{Keno Fischer}

\newcommand{\homework}{\biu{Homework}}
\newcommand{\Mor}{\text{Mor}}
\newcommand{\N}{\mathbb{N}}
\newcommand{\Q}{\mathbb{Q}}
\newcommand{\Z}{\mathbb{Z}}
\newcommand{\R}{\mathbb{R}}
\newcommand{\C}{\mathbb{C}}
\newcommand{\pabs}[1]{\left|\left| #1 \right|\right|_p}
\newcommand{\set}[1]{\left\{#1 \right\}}
\newcommand{\paren}[1]{\left(#1 \right)}
\newcommand{\parens}{\paren}
\newcommand{\tot}[0]{\text{tot}}
%\newcommand{\pabs}[1]{#1}
\begin{document}
\tikzstyle{lattice}=[shape=circle,draw,fill,text=white]
\tikzset{
  % style to apply some styles to each segment of a path
  on each segment/.style={
    decorate,
    decoration={
      show path construction,
      moveto code={},
      lineto code={
        \path [#1]
        (\tikzinputsegmentfirst) -- (\tikzinputsegmentlast);
      },
      curveto code={
        \path [#1] (\tikzinputsegmentfirst)
        .. controls
        (\tikzinputsegmentsupporta) and (\tikzinputsegmentsupportb)
        ..
        (\tikzinputsegmentlast);
      },
      closepath code={
        \path [#1]
        (\tikzinputsegmentfirst) -- (\tikzinputsegmentlast);
      },
    },
  },
  % style to add an arrow in the middle of a path
  end arrow/.style={postaction={decorate,decoration={
        markings,
        mark=at position 1 with {\arrow[#1]{stealth}}
      }}},
}

\author{Keno Fischer, Connor Harris, Alex Lombardi}
\maketitle
\paragraph{Question 1}
\subparagraph*{Part 1-everyone-1}
We will assume that in the cylindrical the speed of light is still equal to $c$ in all initial reference frames. However, as we will see, it is not necessarily true anymore that any two initial reference frames are identical.  \par 
In this universe, sending out a light pulse simultaneously will result in it coming back from both the $+x$ and the $-x$ direction, after two different time intervals $T_+$ and $T_-$. It is easy to see conceptually how such an asymmetry can arise in a cylindrical universe. Suppose our base ship is moving at some velocity $\beta$ in the $+x$ direction and sends out a light pulse at some time $t_0$. Since $c$ is finite, the ship will continue to travel in the $+x$ direction while the light travels all the way around. However, because of that, the light that was sent in the $-x$ direction will meet the ship earlier than the light sent into the $+x$ direction (since it had to travel less and both light pulses move at the same speed). \par 
An interesting and very consequential consequence of this is that the second postulate of special relativity no longer holds in its full generality as the values of $T_+$ and $T_-$ will be different for every reference frame (see the argument for why such a discrepancy may exist in the first place for a reasoning why this should be the case). In particular, there exists one very special frame in which $T_+=T_-$. This frame could easily be called the total rest frame of the universe, since the asymmetry in $T_+$ and $T_-$ can be easily explained by objects moving relative to this reference frame. \par 
Let us now consider the total rest frame and an object traveling at some speed $\beta$ therein. Then we have,
\[ T_+ = \lambda_c(1+\beta) \]
\[ T_- = \lambda_c(1-\beta) \]
or equivalently
\[ \beta = \frac{T_+-T_-}{T_++T_-} \]
Note however that these are times in the total rest frame. 
Now, the invariant interval between the emittance of the light pulse in the $-x$ direction and the reception of the light from the $x$ direction, we have
\[ \Delta s_-^2 = \lambda_c^2(1-\beta)^2 - \lambda_c^2\beta^2 = \lambda_c^2 - 2\beta\lambda_c^2 \]
and for the other invariant interval, we have 
\[ \Delta s_+^2 = \lambda_c^2(1+\beta)^2 - \lambda_c^2\beta^2 = \lambda_c^2 + 2\beta\lambda_c^2 \]
Note that $\Delta s_+^2$ and $\Delta s_+^2$ are just just $(T'_+)^2$ and $(T'_-)^2$ in the $\beta$ frame (i.e. the delays we measure), Thus
\[ \frac{(T'_+)^2 - (T'_-)^2}{(T'_+)^2 + (T'_-)^2} = \frac{4\beta\lambda_c^2}{2\lambda_c^2} = 2\beta \]
So we may solve easily solve for $\beta$ in terms of given quantities. \par 
We can now more easily determine which of the two twins should be older: The one with the larger relative velocity with respect to the total rest frame.
So suppose our spaceship in the $\beta$ frame launches a probe at relative velocity $v$. Then by velocity addition, the velocity with respect to the total rest frame is
\[ u=\frac{\beta+v}{1+v\beta} \]
 \par 
 So suppose that the base ship sends away the probe at a spacetime event given by coordinates
 $(t,x,y,z) = (0,0,0,0)$. This coordinate will be the same in all frames. Now consider the situation in the total rest frame: The base ship moves into the $+x$ direction at a speed $\beta$ and the probe moves at velocity $u$ (though not necessarily in the $-x$ direction, depending on the relative size of $\beta$ and $v$). Then we know that
 \[ t\beta = \lambda_c + tu \]
 So 
 \[ t = \frac{\lambda_c}{\beta - u} = \frac{\lambda_c (1+v\beta)}{v(\beta^2-1)}\]
 in the total rest frame. The question of how much time has passed in each of the individual rest frames reduces to a simple time dilation calculation:
 \[ t_{base}' = \frac{t}{\gamma} = t\sqrt{1-\beta^2} = \frac{\lambda_c (1+v\beta) \sqrt{1-\beta^2}}{v(\beta^2-1)} \]
\[ t_{probe}' = \frac{t}{\gamma} = t\sqrt{1-u^2}
= \frac{\lambda_c (1+v\beta) \sqrt{1-\left(\frac{\beta+v}{1+v\beta}\right)^2}}{v(\beta^2-1)} \]
\subparagraph{Remark}
The following observation was made in an attempt to solve the problem above, before the way outlined above was found and though it is not used in the argument above any more, it is nevertheless interesting to see another surprising property of this universe:\par
To do some more quantitative analysis, we may consider the strange new universe to be a restriction of regular 4-dimensional spacetime,  by taking one of the coordinates ($x$) modulo the circumference of the universe, i.e. we identify all points $(t,x \mod \lambda_c ,y,z)$ where $\lambda_c$ is the circumference of the universe (think of it as unrolling the cylinder and glueing infinitely many copies of it together). Note that the events of the base ship at some $t$ are simultaneous at all $x+n\lambda_c$, for $n\in \Z$. Consider thus the image of a spacetime coordinate $(t, x, y, z)$ under a pure Lorentz boost by some velocity $\beta$ in the $x$-direction:
\[
(ct',x',y',z') = (\gamma(ct-\beta x),\gamma(x - \beta t),y,z) 
\]
but now any of the points $(t,x \mod \lambda_c ,y,z)$ refers to the same spacetime event, and thus we get the following equivalence to describe points in the boosted frame
\[ (ct',x',y',z') \equiv (\gamma(ct-\beta (x+n\lambda_c)),\gamma(x+n\lambda_c - \beta t),y,z) = (ct'-\gamma\beta n \lambda_c,x'+\gamma n \lambda_c,y',z') \]
for $n\in\Z$. This is true for all $v$, but let us choose it in such a way that the resulting frame will be the total rest frame. The result is very, very interesting, for the following reason: The same space time event happens again and again at different times and at different locations (as seen from the total rest frame frame). 

 
\begin{comment}
We will assume that in the cylindrical the speed of light is still equal to $c$ in all initial reference frames. However, as we will see, it is not necessarily true anymore that the any two initial reference are identical.  \par 
In this universe, sending out a light pulse simultaneously will result in it coming back from both the $+x$ and the $-x$ direction, after two different time intervals $T_+$ and $T_-$. It is easy to see conceptually how such an asymmetry can arise in a cylindrical universe. Suppose our base ship is moving at some velocity $u$ in the $+x$ direction and sends out a light pulse at some time $t_0$. Since $c$ is finite, the ship will continue to travel in the $+x$ direction while the light travels all the way around. However, because of that, the light that was sent in the $-x$ direction will meet the ship earlier than the light sent into the $+x$ direction (since it had to travel less and both light pulses move at the same speed). \par 
A perhaps easier and more intuitive way to think about this is to recall the use of simultaneity of events in inertial reference frames: Given some local spacetime event in our initial reference frame, all those events are simultaneous which would be reached by a light pulse emitted directly at the original spacetime event. Thus at any given instant, our ship is simultaneous with a copy of itself, whose internal clock has already advanced by $T_+$ and is located at a distance of $T_+c$ in the $+x$ direction and another copy of itself whose internal clock has advanced by $T_-$ located a distance of $T_-$ in the $-x$ direction. \par 
The analysis of the twin paradox becomes now rather routine: Suppose we are sending out a probe in the $+x$ direction. Then, to reach the ship in the positive $x$ direction, the probe has to move a distance of $T_+c$. Due to length contraction, according to a clock on the probe, the journey takes a total time of \[t=\frac{T_+c}{v\gamma} = T_+\sqrt{c^2/v^2-1}\], on the other hand, seen from the copy of the ship at $x=T_+x$, the probe is emitted by the ship in the middle, when the internal clock of the "copy" has already advanced by $T_+$. Thus the time as seen in the frame of the ship is given by
\[ t_s = T_+(1+\frac{c}{v}) \]
\end{comment}

\subparagraph*{Part 1a)}
Consider the Lagrangian of a particle in this universe (in units where $c=1$):
\[ \Lagr = -m\sqrt{1-\dot{x}^2-\dot{y^2}-\dot{z^2}} - V(x,y,z)  \]
where appropriate boundary conditions are imposed in $V(x,y,z)$.
In the standard Lagrangian on four dimensional spacetime, we have four kinds of symmetries with the associated conserved quantities:
\begin{itemize}
\item Translation in space and time: Energy-Momentum 4-vector (Energy=translation in time, Momentum=translation is space).
\item Rotation in 3-space: Angular momentum
\item Lorentz Boost: Rest mass (i.e. Minkowski inner product of the Energy-Momentum 4-vector)
\end{itemize}
The problem asks us for the first two to examine whether or not there exists a symmetry that leads to a conserved quantity and if not to find a counter example. So let's go through both cases:
For the first category, nothing changes from the classical analysis (with the same condition for conservation of energy and momentum imposed on $V$ as in the classical case). Thus these quantities are conserved by Noether's theorem. This Lagrangian is not, however, symmetric under rotation in 3-space. If we are originally at position $y=2\lambda_c$, we may find ourselves at position $y=\lambda_c$ after a rotation of $2\pi$ and thus it may not be conserved. \par
\emph{Angular momentum is not conserved.}  For a counterexample, restrict our consideration to the $xy$-``plane.''  Take a fixed reference point at coordinates $(x_0, y_0)$, and consider a solitary particle at rest at the coordinates $(x_0, y_0 + h)$.  We can define the notion of distance consistently for long slices in the $y$-direction with arbitrarily small $x$-dimension (that is, for essentially flat parts of the $xy$-space).

Now suppose that some event internal to the particle splits it into two particles of equal mass that travel without substantial interaction.  One half-particle travels on a helical path to $(x_0, y_0)$, and the other half-particle travels on an opposing helical path with the same velocity to $(x_0, y_0 + 2h)$.  Immediately after the reaction, the angular momentum of the system is infinitesimally close to zero, as the two particles have the same distance from the reference point and form equal and opposite angles with it.  However, when the one half-particle travels through the reference point, thus having zero angular momentum, the other half-particle, still travelling at an angle to the $y$-axis does not.  Thus, the two-particle system evolves from zero to non-zero angular momentum.

\paragraph{Question 2}
\subparagraph{2-Everyone}
First, let us determine $x_1$ and $x_2$ in terms of $x$; manipulating the given equations,
$$x_1(t)(\f{m_2}{m_1+m_2})-x_2(t)(\f{m_2}{m_1+m_2})=\f{m_2}{m_1+m_2} C\sin(\om_d t)$$
Adding this to the second equation gives us
$$x_1(t)=x(t)+\f{m_2}{m_1+m_2} C\sin(\om_d t)$$
Similarly, we have the other relation
$$x_2(t)=x(t)-\f{m_1}{m_1+m_2}C\sin(\om_d t)$$
Now, we calculate the Lagrangian in terms of $x_1$ and $x_2$ and then convert
$$L=T-V=\f 1 2 m_1 \dot{x_1}^2+\f 1 2 m_2 \dot{x_2}^2-(\f 1 2 K_1 x_1^2+\f 1 2 K_2 x_2^2)$$
$$L=\f 1 2 m_1(\dot x(t)+\f{m_2}{m_1+m_2} C\om_d \cos(\om_d t))^2+\f 1 2 m_2(\dot x(t) -\f{m_1}{m_1+m_2} C\om_d \cos(\om_d t))^2$$
$$-(\f 1 2 K_1 (x(t)+\f{m_2}{m_1+m_2} C\sin(\om_d t))^2+\f 1 2 K_2 ( x(t)-\f{m_1}{m_1+m_2} C\sin(\om_d t))^2)$$
Next, we set up the Euler-Lagrange equation.
$$\pdif L x=-K_1 (x(t)+\f{m_2}{m_1+m_2}C\sin(\om_d t))-K_2 (x(t)-\f{m_1}{m_1+m_2} C\sin(\om_d t))$$
$$=-(K_1+K_2) x(t)-\f{K_1 m_2-K_2 m_1}{m_1+m_2}C \sin(\om_d t)$$
$$\pdif L {\dot x} = m_1 (\dot x(t)+\f{m_2}{m_1+m_2} C \om_d \cos(\om_d t))+m_2  (\dot x(t)-\f{m_1}{m_1+m_2} C \om_d \cos(\om_d t))$$
$$= (m_1+m_2) \dot x(t)$$
Our Euler-Lagrange equation is then
$$\ddot x(t) (m_1+m_2)+ x(t) (K_1+K_2)=\f{K_2 m_1-K_1 m_2}{m_1+m_2} C \sin(\om_d t)$$
The corresponding homogeneous equation has the general solution
$$x(t)= a \sin(\om_d t)+b\cos(\om_d t)$$
for $\om_d$ as defined in the prompt.  Therefore, to solve for the general solution to this differential equation, we only need to find a single solution.  Suppose we have an ansatz of
$$x(t)=A t\cos(\om_d t)$$
Then, 
$$x'(t)= A\cos(\om_d t)-A\om_d t\sin(\om_d t)$$
$$x''(t)=-2A\om_d \sin(\om_d t)-A\om_d^2 t \cos(\om_d t)$$
Plugging this into the equation from earlier, we get
$$(m_1+m_2)(-2A\om_d \sin(\om_d t)-A\om_d^2 t \cos(\om_d t))$$
$$+(K_1+K_2)( A t \cos(\om_d t))=\f{K_2 m_1-K_1 m_2}{m_1+m_2} C \sin(\om_d t)$$
We know that $\om_d^2 = \f{K_1+K_2}{m_1+m_2}$, so this equation is equivalent to
$$(m_1+m_2)(-2A\om_d \sin(\om_d t))=\f{K_2 m_1-K_1 m_2}{m_1+m_2} C \sin(\om_d t)$$
which is itself equivalent to the condition that
$$A=C\f{K_1 m_2-K_2 m_1}{2\om_d(m_1+m_2)^2}=C\f{K_1 m_2-K_2 m_1}{2(K_1+K_2)^{\f 1 2}(m_1+m_2)^{\f 3 2}}$$
Hence, our general solution to the differential equation is
$$x(t)=A t\cos(\om_d t)+a \cos(\om_d t)+b\sin(\om_d t)$$
where $A$ is the $A$ we just calculated, and $a$ and $b$ are not fixed.  Plugging in our first initial condition ($x(0)=0$) we get that $a=0$.  Therefore, $x(t)=At\cos(\om_d t)+b\sin(\om_d t)\rightarrow \dot x(t)=-At \om_d \sin(\om_d t)+ A\cos(\om_d t)+b\om_d\cos(\om_d t)$.  Plugging in our initial condition $\dot x(0)=0$ then tells us that
$$b=\f{A}{\om_d}=C\f{K_1 m_2-K_2 m_1}{2(K_1+K_2)(m_1+m_2)}$$
Now, we have completely described $x(t)$.  Our earlier equations relating $x_1$ and $x_2$ to $x$ still hold, so given our description of $x(t)$, we find that (for our calculated values of $A$ and $b$)
$$x_1(t)=At \cos(\om_d t)+b\sin(\om_d t)+\f{m_2}{m_1+m_2} C\sin(\om_d t)$$
$$x_2(t)=At\cos(\om_d t)+b\sin(\om_d t)-\f{m_1}{m_1+m_2}C\sin(\om_d t)$$
\subparagraph{Part a}  First, we relate $x_1$ and $x_2$:
$$x_2(t)= x_1(t)-C\sin(\om_d) t$$
Then, we compute the Lagrangian:
$$L=T-V= \f 1 2 (m_1 \dot x_1^2 + m_2 (\dot x_1 - C\om_d \cos(\om_d t))^2)-\f 1 2 (K_1 x_1^2+K_2 (x_1-C \sin(\om_d t)^2))$$
We then compute
$$\pdif L {\dot x_1}= m_1 \dot x_1 + m_2 (\dot x_1 - C \om_d \cos(\om_d t))$$
$$\pdif L {x_1}=-(K_1 x_1+K_2(x_1-C\sin (\om_d t)))$$
The Euler-Lagrange equation is then 
$$  (m_1+m_2)\ddot x_1 + C m_2 \om_d^2 \cos(\om_d t)=-((K_1+K_2) x_1 - C K_2 \sin(\om_d t))$$
\subparagraph{Part b}  The relation between $x_1$ and $x_2$ we will be using is
$$x_1(t)=x_2(t) + C\sin(\om_d t)$$
Therefore, the Lagrangian for the system is 
$$L=T-V= \f 1 2 (m_2 \dot x_2^2 + m_1 (\dot x_2 + C\om_d \cos(\om_d t))^2)-\f 1 2 (K_2 x_2^2+K_1 (x_2+C \sin(\om_d t)^2))$$
We then compute
$$\pdif L {\dot x_2}= m_2 \dot x_2 + m_1 (\dot x_2+ C \om_d \cos(\om_d t))$$
$$\pdif L {x_2}=-(K_2 x_2+K_1(x_2+C\sin (\om_d t)))$$
The Euler-Lagrange equation is then
$$ (m_1+m_2) \ddot x_2-m_1 C \om_d^2 \sin(\om_d t)=-((K_1+K_2) x_2+C K_1 \sin(\om_d t))$$

\paragraph{Question 3}
\subparagraph{Part 1}
We $\parens{will}$ assume that Shaun starts in a horizontal position (i.e. the board is vertical) and that his center of mass is located at a height of $R$ above the bottom of the circle. Such an assumption is reasonable, because any other initial configuration would only add a constant term of additional energy which would not qualitatively affect the analysis below.

Thus by the initial conditions we have that Shaun has initial energies of
\[ KE_0 = 0 \]
\[ PE_0 = mgR \]
Now, note that we cannot just use the conservation of energy directly as there is work being done in the instance that Shaun rises and we do not now (yet) what the magnitude of this work is. To find said work, consider the angular momentum about the center of the circle (which must be equal before and after the instantaneous rise since the force is along the axis between the origin of the circle and Shaun's center of mass):
\[ \vec{L_A} = \vec r \times \vec p = rmv \hat{z} \]
where $\hat{z}$ is pointing into the plane. \par
Now, we may easily find the velocity at the instantaneous moment before the rise $v_A$ by analyzing the energy (and thereby using the fact that energy is conserved in the absence of friction.
\[ v_A=\sqrt{2\frac{KE_A}{m}} = \sqrt{2\frac{PE_0-PE_A}{m}} = \sqrt{2g(R-(l-h))}  \]
We now that angular momentum is conserved in this instant and thus the speed afterwards $v_B$ is given by
\[ v_B = \frac{r_A}{r_B} v_A = \frac{l-h}{R-(l-h)} v_A = \left(1+\frac{h}{R-l}\right) v_a \]
So we have
\[ KE_B = KE_A\left(1+\frac{h}{R-l}\right)^2 \]
\[ PE_B = mgl \]
Now, we may find the distance of Shaun ascend over the half pipe by comparing this new total energy with the potential energy at his highest ascend, so we get, 
\[ d=\frac{KE_B+PE_B}{mg}-R = \left(R-l+h\right)\left(1+\frac{h}{R-l}\right)^2+l-R\]
To first order in small dimensionless quantities, 
\[ \left(R-l+h\right)\left(1+\frac{h}{R-l}\right)^2 = \left(R-l+h\right) \left(1+\frac{2h}{R-l}+\frac{h^2}{(R-l)^2}\right) \approx (R-l+h)+(2h) \]
so we get
\[ d = l-R + R - l + 3h =3h \]
\subparagraph{Part 2}
It is indeed possible for Shaun to increase his maximum final height in this situation - his strategy should be as follows:
\\\\ (1) Instantaneously raise his center of mass to its maximum height as he enters the straightaway
\\ (2) Lower his center of mass over the straightaway so that it it is at its minimum height again by the end of it.
\\ (3) Instantaneously raise his center of mass to its maximum height again as he leaves the straightaway.
\\\\ Step (1) will have the same effect that it did in 3-Everyone-1 - that is, given that $v_b$ is Shaun's velocity right before he raises his center of mass, his new velocity will be $v_n=v_b (1+\f{h}{R-l})$.  
\\\\ As Shaun travels along the straightaway, the net force acting on his center of mass is entirely in the $y$-direction.  This is because the only forces acting on his center of mass are gravity, the normal force, and the force he exerts on his center of mass to move it downwards.  These are all in the $y$-direction (in fact, the first two cancel each other out), so the net force in the $x$ direction is zero.  Therefore, Shaun's velocity in the $x$ direction does not change over the entirety of step (2).  
\\\\ Step (3) can be treated exactly the same as step (1) - we calculate the angular momentum $L$ with respect to the center of the new circle about which Shaun is to rotate.  That is, 
$$L=\vec r\times \vec p$$
Where $\vec p=m\vec v=m(\vec v_n \hat x+\vec v_y \hat y)$ and $v_n$ is as before (due to our work on step (2)).  Well, $\vec r=(R-l+h)\hat y$, so $\vec r\times \vec p=\vec r\times m\vec v_n \hat x$.  Then, the conservation of angular momentum equation tells us that ${v_f}_x=v_n (1+\f{h}{R-l})$.  Furthermore, as Shaun cannot extend his center of mass any higher while remaining on the ground, we find that ${v_f}_y=0\rightarrow v_f=v_n(1+\f{h}{R-l})$.  Hence, we get that
$$v_f=\sqrt{2g(R-(l-h))}(1+\f{h}{R-l})^2$$
where $v_f$ is Shaun's velocity right as he begins the second quarter circle.  Then, his total energy for the remainder of his motion is
$$E=mg(R-(l-h)) (1+\f{h}{R-l})^4+mgl$$
and hence his maximum height occurs at
$$h_{\max}=(R-(l-h))(1+\f{h}{R-l}^4)\approx (R-(l-h))(1+\f{4h}{R-l})+l$$
after first-order approximation.  This then simplifies to
$$h_{\max}\approx R+h+\f{4h(R-l+h)}{R-l}$$
which, under first order approximation again, simplifies to
$$h_{\max}\approx R+h+4h=R+5h$$
giving us a final answer of $5h$.
\subparagraph{Part 3a)}
Since there is no friction, we know that the only forces acting on Shaun are the normal force and gravity. Now, consider the situation in which Shaun's center of mass is located some distance $z$ below the top of the half pipe. Then we know that
$$z=(R-l)\cos(\alpha)$$
Where $\vec r=R\cos(\f \pi 2-\alpha)\hat x-R\sin(\f \pi 2-\alpha)\hat y=R\sin(\alpha)\hat x-R\cos(\alpha)\hat y$.  Define $\vec r_0$ to be the position of the front wheel and $\vec r_1$ to be the position of the back wheel.  Then, we can approximate 
$$\vec r_0=R(\sin(\alpha), -\cos(\alpha))+\f{w}{2}(-\cos(\alpha), -\sin(\alpha))$$
$$\vec r_1=R(\sin(\alpha), -\cos(\alpha))-\f w 2(-\cos(\alpha), -\sin(\alpha))$$
The first term of these two sums is the vector to the edge of the circle that goes through the center of mass, and the second terms approximate the board to be flat (on the tangent) on the circle.  Define $\vec c$ to be the position of the center of mass, and we get that $\vec c=(R-l)(\sin(\alpha), -\cos(\alpha))$.  Thus,
$$\vec c-\vec r_0=-l(\sin(\alpha), -\cos(\alpha))-\f{w}{2}(-\cos(\alpha), -\sin(\alpha))$$
$$\rightarrow \vec c-\vec r_0=l(-\sin(\alpha), \cos(\alpha))+\f w 2(\cos(\alpha), \sin(\alpha))$$
and similarly
$$\vec c-\vec r_1=l(-\sin(\alpha), \cos(\alpha))-\f w 2(\cos(\alpha), \sin(\alpha))$$
This is necessary to calculate in order to apply the formula
$$\sum \tau_i=\dif L t+m(\vec c-\vec r)\times \ddot {\vec r}$$
This formula allows us to calculate the combined torque of gravity and the normal force acting on each of the wheels.  Well, we first need to calculate
$$\dot{\vec r}_0=R(\cos(\alpha)\dif \alpha t, \sin(\alpha)\dif \alpha t)+\f w 2(\sin(\alpha)\dif \alpha t, -\cos(\alpha)\dif \alpha t)$$
$$\dot{\vec r}_1=R(\cos(\alpha)\dif \alpha t, \sin(\alpha)\dif \alpha t)-\f w 2 (\sin(\alpha)\dif \alpha t, -\cos(\alpha)\dif \alpha t)$$
Using the fact that $\dif {(\f \pi 2-\alpha)}t=\f{|\vec {\dot r}|}{R}$, we get that $\dif \alpha t=-\f {v} R$.  Furthermore, using conservation of energy, we have that
$$\f 1 2 m v^2=mgz\rightarrow v=\sqrt {2gz}$$
Thus, we have $\dif \alpha t=-\f{\sqrt {2gz}}R$.  Then, we can calculate
$$\ddot{\vec r}_0=R\dif \alpha t(-\sin(\alpha)\dif \alpha t, \cos(\alpha) \dif \alpha t)+R \dif{\dif \alpha t}t(\cos(\alpha), \sin(\alpha))$$
$$+\f w 2 \dif \alpha t (\cos(\alpha) \dif \alpha t, \sin(\alpha) \dif \alpha t)+\f w 2 \dif{\dif \alpha t}t(\sin(\alpha), -\cos(\alpha))$$
With the formula for $r_1$ being the same (with a $-\f w 2$ second term).  Now, we can calculate
$$m(\vec c-\vec r_0)\times \ddot {\vec r}_0=m[-l R \dif{\dif \alpha t}t-\f{lw}{2}(\dif \alpha t)^2+\f {Rw}{2}(\dif \alpha t)^2-(\f w 2)^2 \dif{\dif \alpha t}t]\hat z$$
To completely evaluate this, we need to calculate $\dif {\dif \alpha t}t$, which is determined by calculating $\dif z t$.  Well, 
$$\cos(\alpha)=\f{z}{R-l}$$
Hence, $$\dif z t=-(R-l)\sin(\alpha) \dif \alpha t=\f{\sqrt{(R-l)^2-z^2}\sqrt{2gz}(R-l)}{R(R-l)}=\f{\sqrt{(R-l)^2-z^2}\sqrt{2gz}}{R}$$
Then, we can calculate
$$\dif {\dif \alpha t}t=\f {2g \dif z t}{R\sqrt{2gz}}=\f {2g \sqrt{(R-l)^2-z^2}}{R^2}$$
As we will see in later calculations, we will need to divide $m(\vec c-\vec r_0)\times \ddot {\vec r}_0$ by $w$ and consider all significant terms.  In that case, the second and fourth terms of the expression I gave will have terms of $\f{l}{R}$ or $\f{w}{R}$, and hence are insignificant.  Therefore, we can make the approximation
$$m(\vec c-\vec r_0)\times \ddot {\vec r}_0\approx m[\f{-2lg\sqrt{(R-l)^2-z^2}}{R}+\f{wgz}{R}]\hat z$$
Through the same calculations, we find
$$m(\vec c-\vec r_1)\times \ddot{\vec r}_1\approx m(\f{-2lg\sqrt{(R-l)^2-z^2}}{R}-\f {wgz}R)\hat z$$
We can also calculate the angular momentum of Shaun's center of mass with respect to $r_0$:
$$\vec p=m(\dot{\vec c}-\dot{\vec r}_0)=m(l\dif \alpha t(-\cos(\alpha), -\sin(\alpha))+\f w 2 \dif \alpha t(-\sin(\alpha), \cos(\alpha)))$$
$$\vec L=(\vec c-\vec r_0)\times \vec p=m\dif \alpha t(l^2+(\f w 2)^2)$$
Thus, 
$$\dif{\vec L}t=\f{2mg\sqrt{(R-l)^2-z^2}(l^2+(\f w 2)^2)}{R^2}$$
Similarly, we need to take the significant terms of $\dif{\vec L}t$ when divided by $w$; however, in this case, there will be a factor of $\f{l+\f{w^2}{l}}{R}$ in the result, so this entire thing is in fact zero!  $\dif {\vec L}{t}\approx 0$ (and this holds for both $r_0$ and $r_1$ for the same reasons).  
Therefore, our two equations reduce to
$$\sum \tau_i\approx [\f{-2mlg\sqrt{(R-l)^2-z^2}}{R}+\f{mwgz}{R}]\hat z$$
$$\sum \tau_i\approx [\f{-2mlg\sqrt{(R-l)^2-z^2}}{R}-\f{mwgz}{R}]\hat z$$
Now, there are only two torques to consider that are acting on the first wheel - torque due to gravity, and torque due to the normal force $\vec F_n$ of the \textit{second} wheel acting on the half-pipe.  Well, $\tau_n=w F_n \hat z$, as the vector from the front wheel's position to the back wheel's position and $\vec F_n$ are perpendicular to each other.  As for the normal force on the first wheel, we get that $\tau_n=-w F_n \hat z$, because the displacement vector from the back wheel's position to the front wheel's position is in the opposite direction.
\\\\ The torque due to gravity is also calculable, given everything we've done already.  We know that $F_g=(0, -mg)$ and that our displacement vector is $\vec r_0-c=-l(-\sin(\alpha), \cos(\alpha))-\f w 2(\cos(\alpha), \sin(\alpha))$.  Therefore, 
$$\tau_g=-mg(l\sin(\alpha)-\f w 2 \cos(\alpha))\hat z=-mg(\f{l\sqrt{(R-l)^2-z^2}-\f {wz}{2}}{R-l})\hat z$$
For $r_1$, we get the similar equation
$$\tau_g=-mg\f{l\sqrt{(R-l)^2-z^2}+\f {wz}2}{R-l} \hat z$$
Finally, we have everything that we need to plug into our master equation!  For the first wheel:
$$w F_n \hat z\approx  m[\f{-2lg\sqrt{(R-l)^2-z^2}}{R}+\f{wgz}{R}]\hat z+mg(\f{l\sqrt{(R-l)^2-z^2}-\f {wz}{2}}{R-l})\hat z$$
Thus, with some approximation of $\f {lz} R\approx \f {lz}{R-l}$, we get
$$F_n\approx mg(-2l\sqrt{1-(\f z {R-l})^2}+ \f {wz}{R}+l\sqrt{1-(\f z {R-l})^2}-\f {wz}{2R})\approx mg(-l\sqrt{1-(\f z R)^2}+\f{wz}{2R})$$
Furthermore, we know that the direction of $F_n$ is in the same direction as $\vec r_0$; therefore, $\vec F_n=F_n\f{-\vec r_0}{r_0}$.  There are two forces acting directly on the second wheel; the normal force $\vec F_n$ that we just calculated, and $F_2$, a component of the gravity force acting directly on it.  Using Newton's second law, we get
$$\vec F_2+\vec F_n=m \ddot r_1$$
And hence the two forces acting on the second wheel are $\vec F_n$ as calculated and $\vec F_2=m\ddot r_1-\vec F_n$.
\\\\ For the second wheel:
$$w F_n \hat z\approx -mg\f{l\sqrt{(R-l)^2-z^2}+\f {wz}2}{R-l} \hat z-m(\f{-2lg\sqrt{(R-l)^2-z^2}}{R}-\f {wgz}R)\hat z$$
Through the same approximation methods, we arrive at our conclusion:
$$F_n \approx mg(l\sqrt{1-(\f z {R-l})^2}+\f{wz}{2R})$$
Then, we find that the two forces acting on the first wheel are $\vec F_n=F_n \f{-\vec r_0}{r_0}$ and $\vec F_1=m\ddot{\vec r}_0-\vec F_n$.  Note that (as a coherence check) when $z\approx R-l$ (which is when Shaun reaches the bottom of the half-pipe), we find that the two different normal forces are equal, with $F_n=\f {mg}2$, which means that the weight of the center of mass is evenly distributed on the two wheels at this point in time (which is in line with our intuition).  

\paragraph{Question 4.}

\subparagraph{Part 1.}  Moment-of-inertia tensors about the same point (in this case, the shared center of mass) simply add, so the total moment-of-inertia tensor is $$\overleftrightarrow{I}
\eqdef \overleftrightarrow{I^o} + \overleftrightarrow{I^i} =
\begin{bmatrix}
I_\bot^i + I_\bot^o & 0 & 0 \\
0 & I_\bot^i + I_\bot^o & 0 \\
0 & 0 & I_3^i + I_3^o
\end{bmatrix}$$

\subparagraph{Part 2.}

Though $\vec{L} \eqdef \vec{L}^i + \vec{L}^o$ stays constant in a fixed space frame, the angular momenta $\vec{L}^i$ and $\vec{L}^o$ of the parts of the station do not necessarily stay constant, because each part of the station exerts a torque on the other part.  Fix a coordinate frame in space such that $\vec{L}$ points parallel to $\hat{z}$, $\hat{y}$ points parallel to $\hat{e}_2$, and the positive $xz$-quadrant contains $\vec{\omega}$. When the two parts separate, $\vec{L}^i$ and $\vec{L}^o$ thus define new axes of precession for the parts of the spaceship.  The separated parts begin to precess about these new axes, instantaneously changing the angles and possibly directions of precession.  The inner part precesses with angular velocity $\tilde{\omega}^i = L^i/I_\bot^i$ and the outer part precesses with angular velocity $\tilde{\omega}^o = L^o/I_{\bot}^o$, where $L^i$ without superscript is the magnitude of $\vec{L^i}$, and likewise for $L^o$.

\subparagraph{Part 3.}

The axes of symmetry $\hat{e}_3^i$ and $\hat{e}_3^o$ precess in circles with one point of exterior tangency at the breakup point and no other points of intersection.  The two parts of the space station will therefore align again after each component has completed an integer number of revolutions, which requires the quantity $\tilde{\omega}^i / \tilde{\omega}^o$ to be rational.

For the remainder of this problem, let $\omega_3$ and $\omega_\bot$ be the components of $\vec{\omega}$ along $\hat{e}_3$ and $\hat{e}_1$, respectively.  We can compute the magnitudes of $\vec{L}^i$ and $\vec{L}^o$ just as well in the body frame as in the fixed frame.  In the body frame, we have $$\vec{L}^i = \overleftrightarrow{I^i}\vec{\omega} = \omega_\bot I^i_\bot \hat{e}_1 + \omega_3 I_3 \hat{e}_3$$ and thus $$L^i = \sqrt{\omega_\bot^2 (I_\bot^i)^2 + \omega_3^2 (I_3^i)^2}.$$  The formula for $L^o$ repeats this analysis, swapping superscript $o$ for superscript $i$.  Thus, the axes of symmetry of the two parts of the space station become parallel again only if $$\frac{\tilde{\omega}^i}{\tilde{\omega}^o} = \frac{L^o I^i_\bot}{L^i I^o_\bot} = \frac{I^i_\bot \sqrt{\omega_\bot^2 (I_\bot^o)^2 + \omega_3^2 (I_3^o)^2}}{I^o_\bot \sqrt{\omega_\bot^2 (I_\bot^i)^2 + \omega_3^2 (I_3^i)^2}}$$ is rational.  If we let $\theta$ be the angle formed by $\vec{\omega}$ in the $\hat{e}_1 \hat{e}_3$ plane (such that $\omega_3 = \omega \sin \theta$ and $\omega_\bot = \omega \cos \theta$), then we have this perhaps nicer-looking version of the same quantity: $$\frac{I^i_\bot \sqrt{(I^o_\bot )^2 \cos^2 \theta + (I^o_3)^2 \sin^2 \theta}}{I^o_\bot \sqrt{(I^i_\bot)^2 \cos^2 \theta + (I^i_3)^2 \sin^2 \theta}}.$$

\subparagraph{Part 4.}  As the horizontal rods are massless and rigid, any torque exerted on the horizontal rods will be transferred to the inner control center.  It therefore suffices to compute the torque on the control center using Euler's equations.  First, recall the following equations, where $\Omega \eqdef \omega_3 (I_3^t - I_\bot^t)/I_\bot^t$ (the $t$ stands for ``total,'' or both parts of the ship): $$\vec{\omega}(t) = \omega_\bot \cos (\Omega t) \hat{e}_1 + \omega_\bot \sin (\Omega t) \hat{e}_2 + \omega_3 \hat{e}_3$$  $$\dot{\vec{\omega}}(t) = - \Omega \omega_\perp \sin (\Omega t) \hat{e}_1 + \Omega \omega_\perp \cos (\Omega t) \hat{e}_2.$$

Thus, $\vec{\omega}(0) = \omega_\perp  \hat{e}_1 + \omega_3 \hat{e}_3$ and $\dot{\vec{\omega}}(0) = \Omega \omega_\perp \hat{e}_2$.  Now, with $I_1 = I_2 \eqdef I_\perp^i$ and $I_3 \eqdef I_3^i$, we can evaluate each of Euler's equations at $t = 0$ easily:

\begin{itemize}

\item $\tau_1 = I_1 \dot{\omega}_1 + (I_3 - I_2) \omega_3 \omega_2$ evaluates to $\tau_1(0) = 0$, as $\dot{\omega}_1(0) = 0$ and $\omega_2(0) = 0$.

\item $\tau_2 = I_2 \dot{\omega}_2 + (I_1 - I_3) \omega_1 \omega_3$ evaluates to $\tau_2(0) = I_\bot^i \Omega \omega_\bot + (I_\bot^i - I_3^i) \omega_\bot \omega_3$.

\item $\tau_3 = I_3 \dot{\omega}_3 + (I_2 - I_1) \omega_2 \omega_2$ evaluates to $\tau_3(0) = 0$, as $I_1 = I_2$ and $\dot{\omega}_3 = 0$.

We thus get the following expression for torque: $$\boxed{\tau(0) = (I_\bot^i \Omega \omega_\bot + (I_\bot^i - I_3^i) \omega_\bot \omega_3) \hat{e}_2.}$$

\end{itemize}

\begin{comment} To compute the torque, therefore, it suffices to compute $d\vec{L^i}/dt$ for $t = 0$ in the fixed space frame.  We apply the equation $d\vec{L}^i/dt = \partial \vec{L}^i/\partial t + \omega \times \vec{L}^i$.

Consider the first term first.  We have $\vec{\omega} = \omega_\bot \cos (\Omega t + \phi) \hat{e}_1 + \omega_\bot \sin (\Omega t + \phi) \hat{e}_2 + \omega_3 \hat{e}_3$, where $\Omega = \omega_3 (I^i_3 - I^i_\bot)/I^i_\bot$.  Our choice of coordinate system gives us $\phi = 0$, so $\dot{\vec{\omega}}(0) = \omega_\bot \Omega \hat{e}_2$.  As $\vec{L}^i = \overleftrightarrow{I^i} \vec{\omega}$ in the body frame, evaluating $\partial \vec{L}^i/\partial t$ at $t = 0$ gives us $\overleftrightarrow{I^i} \dot{\vec{\omega}}(0) = I^i_\bot \Omega \omega_\bot \hat{e}_2 = (I^i_3 - I^i_\bot) \omega_\bot \omega_3 \hat{e}_2$.

Now consider the second term.  We can take the cross product in the body frame, in which $\vec{L^i} = \overleftrightarrow{I^i} \vec{\omega}$ is still valid.  At $t = 0$, $\vec{\omega} = \omega_\bot \hat{e}_1 + \omega_3 \hat{e}_3$, so $\vec{L^i} = I^i_\bot \omega_\bot \hat{e}_1 + I^i_3 \omega_3 \hat{e}_3$ and $\omega \times \vec{L^i} = - (I^i_3 + I^i_\bot) \omega_\bot \omega_3 \hat{e}_2$.  This gives us our final answer for the torque: $$\boxed{\tau = - 2 I^i_\bot \omega_\bot \omega_e \hat{e}_2.}$$

\end{comment}

\end{document}
