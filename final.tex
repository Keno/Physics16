\title{Physics 16 Final Project}
\documentclass[12pt,letterpaper]{article}

\usepackage{graphicx}
\usepackage{caption}
\usepackage{subcaption}
\usepackage{amsmath} 
\usepackage{amssymb}
\usepackage{ulem}
\usepackage{tikz}
\usepackage{multicol}
\usepackage{verbatim}
\usepackage[left=1in,top=1in,right=1in,bottom=1in,nohead]{geometry}
\usetikzlibrary{decorations.markings}
\usetikzlibrary{decorations.pathreplacing}

\usepackage{amsthm} 
\usepackage{wrapfig}
\usepackage{enumitem}
%\usepackage{enumerate}
\newtheorem{mydef}{Definition}
\newtheorem{example}{Example}
\newtheorem{thrm}{Theorem}
\newtheorem{lemma}{Lemma}
\newtheorem{cor}{Corollary}
\newtheorem{notation}{Notation}
\newtheorem{rem}{Remarks}
\newcommand{\biu}[1]{\underline{\textbf{\textit{#1}}}}
\newcommand{\so}{\Rightarrow}
\newcommand{\Lagr}{\mathcal{L}}
\usepackage[ampersand]{easylist}

\let\oldemptyset\emptyset
\let\emptyset\varnothing

\author{Keno Fischer}

\newcommand{\homework}{\biu{Homework}}
\newcommand{\Mor}{\text{Mor}}
\newcommand{\N}{\mathbb{N}}
\newcommand{\Q}{\mathbb{Q}}
\newcommand{\Z}{\mathbb{Z}}
\newcommand{\R}{\mathbb{R}}
\newcommand{\C}{\mathbb{C}}
\newcommand{\pabs}[1]{\left|\left| #1 \right|\right|_p}
\newcommand{\set}[1]{\left\{#1 \right\}}
\newcommand{\paren}[1]{\left(#1 \right)}
\newcommand{\parens}{\paren}
\newcommand{\tot}[0]{\text{tot}}
%\newcommand{\pabs}[1]{#1}
\begin{document}
\tikzstyle{lattice}=[shape=circle,draw,fill,text=white]
\tikzset{
  % style to apply some styles to each segment of a path
  on each segment/.style={
    decorate,
    decoration={
      show path construction,
      moveto code={},
      lineto code={
        \path [#1]
        (\tikzinputsegmentfirst) -- (\tikzinputsegmentlast);
      },
      curveto code={
        \path [#1] (\tikzinputsegmentfirst)
        .. controls
        (\tikzinputsegmentsupporta) and (\tikzinputsegmentsupportb)
        ..
        (\tikzinputsegmentlast);
      },
      closepath code={
        \path [#1]
        (\tikzinputsegmentfirst) -- (\tikzinputsegmentlast);
      },
    },
  },
  % style to add an arrow in the middle of a path
  end arrow/.style={postaction={decorate,decoration={
        markings,
        mark=at position 1 with {\arrow[#1]{stealth}}
      }}},
}



\paragraph{Question 3 - 1}
We will start by considering Shaun being positioned exactly horizontally (which we may assume since $w << R$in the upper right corner of the halfpipe.

\paragraph{Question 4.}

\subparagraph{Part 1.}  Moment-of-inertia tensors simply add, so the
total moment-of-inertia tensor is $$\overleftrightarrow{I_\tot}
= \overleftrightarrow{I^o} + \overleftrightarrow{I^i} =
\begin{bmatrix}
I_\bot^i + I_\bot^o & 0 & 0 \\
0 & I_\bot^i + I_\bot^0 & 0 \\
0 & 0 & I_3^i + I_3^o
\end{bmatrix}$$


\subparagraph{Part 2.}  The general solution for a free symmetric top
(Morin's equations 9.50 and 9.51) gives us the following general solution for
the motion of the entire space station: $$\vec{\omega}
= \parens{A \cos (\Omega t + \phi), A \sin (\Omega t
+ \phi), \omega_3}$$ $$\vec{L} = \parens{(I_\bot^i + I_\bot^o) A \cos
(\Omega t + \phi), (I_\bot^i + I_\bot^o) A \sin (\Omega t + \phi),
(I_3^o + I_3^i) \omega_3}$$ where $$\Omega = \parens{\frac{(I_3^o + I_3^i) -
(I_\bot^o + I_\bot^i)}{I} \omega_3}$$ and $A$ is determined by the
magnitude of the off-center component of $\vec{\omega}$.  We can set
$\hat{x}, \hat{y}, \hat{z}$ in our coordinate system such that
$\vec{\omega}$ oscillates about the $z$-axis and has a zero
$\hat{y}$-component at $t=0$, allowing us to set $\phi = 0$.  Let $\theta$
be the constant angle between $\omega_3$ and $\hat{z}$; this implies
$\tan \theta = A (I_\bot^o + I_\bot^i)/(I_3^o + I_3^i)$.

If the exterior and interior of the space station are disconnected,
then they will thereafter behave as two independent free symmetric
tops.  As the ratio of $I_\bot$ to $I_3$ differs for each, they will
no longer necessarily precess around the original axis, and not
necessarily with the original speed.

\end{document}
