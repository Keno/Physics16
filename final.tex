\title{Physics 16 Final Project}
\newcommand{\f}{\frac} 
\newcommand{\om}{\omega}
\newcommand{\pdif}[2]{\f{\partial #1}{\partial #2}}
\newcommand{\dif}[2]{\f{d#1}{d#2}}

\documentclass[12pt,letterpaper]{article}

\usepackage{graphicx}
\usepackage{caption}
\usepackage{subcaption}
\usepackage{amsmath} 
\usepackage{amssymb}
\usepackage{ulem}
\usepackage{tikz}
\usepackage{multicol}
\usepackage{verbatim}
\usepackage[left=1in,top=1in,right=1in,bottom=1in,nohead]{geometry}
\usetikzlibrary{decorations.markings}
\usetikzlibrary{decorations.pathreplacing}

\usepackage{amsthm} 
\usepackage{wrapfig}
\usepackage{enumitem}
%\usepackage{enumerate}
\newtheorem{mydef}{Definition}
\newtheorem{example}{Example}
\newtheorem{thrm}{Theorem}
\newtheorem{lemma}{Lemma}
\newtheorem{cor}{Corollary}
\newtheorem{notation}{Notation}
\newtheorem{rem}{Remarks}
\newcommand{\biu}[1]{\underline{\textbf{\textit{#1}}}}
\newcommand{\so}{\Rightarrow}
\newcommand{\Lagr}{\mathcal{L}}
\usepackage[ampersand]{easylist}

\let\oldemptyset\emptyset
\let\emptyset\varnothing

\author{Keno Fischer}

\newcommand{\homework}{\biu{Homework}}
\newcommand{\Mor}{\text{Mor}}
\newcommand{\N}{\mathbb{N}}
\newcommand{\Q}{\mathbb{Q}}
\newcommand{\Z}{\mathbb{Z}}
\newcommand{\R}{\mathbb{R}}
\newcommand{\C}{\mathbb{C}}
\newcommand{\pabs}[1]{\left|\left| #1 \right|\right|_p}
\newcommand{\set}[1]{\left\{#1 \right\}}
\newcommand{\paren}[1]{\left(#1 \right)}
\newcommand{\parens}{\paren}
\newcommand{\tot}[0]{\text{tot}}
%\newcommand{\pabs}[1]{#1}
\begin{document}
\tikzstyle{lattice}=[shape=circle,draw,fill,text=white]
\tikzset{
  % style to apply some styles to each segment of a path
  on each segment/.style={
    decorate,
    decoration={
      show path construction,
      moveto code={},
      lineto code={
        \path [#1]
        (\tikzinputsegmentfirst) -- (\tikzinputsegmentlast);
      },
      curveto code={
        \path [#1] (\tikzinputsegmentfirst)
        .. controls
        (\tikzinputsegmentsupporta) and (\tikzinputsegmentsupportb)
        ..
        (\tikzinputsegmentlast);
      },
      closepath code={
        \path [#1]
        (\tikzinputsegmentfirst) -- (\tikzinputsegmentlast);
      },
    },
  },
  % style to add an arrow in the middle of a path
  end arrow/.style={postaction={decorate,decoration={
        markings,
        mark=at position 1 with {\arrow[#1]{stealth}}
      }}},
}

\textbf{2-everyone)}  First, let us determine $x_1$ and $x_2$ in terms of $x$: Manipulating the given equations,
$$x_1(t)(\f{m_2}{m_1+m_2})-x_2(t)(\f{m_2}{m_1+m_2})=\f{m_2}{m_1+m_2} C\sin(\om_d t)$$
Adding this to the second equation gives us
$$x_1(t)=x(t)+\f{m_2}{m_1+m_2} C\sin(\om_d t)$$
Similarly, we have the other relation
$$x_2(t)=x(t)-\f{m_1}{m_1+m_2}C\sin(\om_d t)$$
Now, we calculate the Lagrangian in terms of $x_1$ and $x_2$ and then convert
$$L=T-V=\f 1 2 m_1 \dot{x_1}^2+\f 1 2 m_2 \dot{x_2}^2-(\f 1 2 K_1 x_1^2+\f 1 2 K_2 x_2^2)$$
$$L=\f 1 2 m_1(\dot x(t)+\f{m_2}{m_1+m_2} C\om_d \cos(\om_d t))^2+\f 1 2 m_2(\dot x(t) -\f{m_1}{m_1+m_2} C\om_d \cos(\om_d t))^2$$
$$-(\f 1 2 K_1 (x(t)+\f{m_2}{m_1+m_2} C\sin(\om_d t))^2+\f 1 2 K_2 ( x(t)-\f{m_1}{m_1+m_2} C\sin(\om_d t))^2)$$
Next, we set up the Euler-Lagrange equation.
$$\pdif L x=-K_1 (x(t)+\f{m_2}{m_1+m_2}C\sin(\om_d t))-K_2 (x(t)-\f{m_1}{m_1+m_2} C\sin(\om_d t))$$
$$=-(K_1+K_2) x(t)-\f{K_1 m_2-K_2 m_1}{m_1+m_2}C \sin(\om_d t)$$
$$\pdif L {\dot x} = m_1 (\dot x(t)+\f{m_2}{m_1+m_2} C \om_d \cos(\om_d t))+m_2  (\dot x(t)-\f{m_1}{m_1+m_2} C \om_d \cos(\om_d t))$$
$$= (m_1+m_2) \dot x(t)$$
Our Euler-Lagrange equation is then
$$\ddot x(t) (m_1+m_2)+ x(t) (K_1+K_2)=\f{K_2 m_1-K_1 m_2}{m_1+m_2} C \sin(\om_d t)$$
The corresponding homogeneous equation has the general solution
$$x(t)= a \sin(\om_d t)+b\cos(\om_d t)$$
for $\om_d$ as defined in the prompt.  Therefore, to solve for the general solution to this differential equation, we only need to find a single solution.  Suppose we have an ansatz of
$$x(t)=A t\cos(\om_d t)$$
Then, 
$$x'(t)= A\cos(\om_d t)-A\om_d t\sin(\om_d t)$$
$$x''(t)=-2A\om_d \sin(\om_d t)-A\om_d^2 t \cos(\om_d t)$$
Plugging this into the equation from earlier, we get
$$(m_1+m_2)(-2A\om_d \sin(\om_d t)-A\om_d^2 t \cos(\om_d t))$$
$$+(K_1+K_2)( A t \cos(\om_d t))=\f{K_2 m_1-K_1 m_2}{m_1+m_2} C \sin(\om_d t)$$
We know that $\om_d^2 = \f{K_1+K_2}{m_1+m_2}$, so this equation is equivalent to
$$(m_1+m_2)(-2A\om_d \sin(\om_d t))=\f{K_2 m_1-K_1 m_2}{m_1+m_2} C \sin(\om_d t)$$
which is itself equivalent to the condition that
$$A=C\f{K_1 m_2-K_2 m_1}{2\om_d(m_1+m_2)^2}=C\f{K_1 m_2-K_2 m_1}{2(K_1+K_2)^{\f 1 2}(m_1+m_2)^{\f 3 2}}$$
Hence, our general solution to the differential equation is
$$x(t)=A t\cos(\om_d t)+a \cos(\om_d t)+b\sin(\om_d t)$$
where $A$ is the $A$ we just calculated, and $a$ and $b$ are not fixed.  Plugging in our first initial condition ($x(0)=0$) we get that $a=0$.  Therefore, $x(t)=At\cos(\om_d t)+b\sin(\om_d t)\rightarrow \dot x(t)=-At \om_d \sin(\om_d t)+ A\cos(\om_d t)+b\om_d\cos(\om_d t)$.  Plugging in our initial condition $\dot x(0)=0$ then tells us that
$$b=\f{A}{\om_d}=C\f{K_1 m_2-K_2 m_1}{2(K_1+K_2)(m_1+m_2)}$$
Now, we have completely described $x(t)$.  Our earlier equations relating $x_1$ and $x_2$ to $x$ still hold, so given our description of $x(t)$, we find that (for our calculated values of $A$ and $b$)
$$x_1(t)=At \cos(\om_d t)+b\sin(\om_d t)+\f{m_2}{m_1+m_2} C\sin(\om_d t)$$
$$x_2(t)=At\cos(\om_d t)+b\sin(\om_d t)-\f{m_1}{m_1+m_2}C\sin(\om_d t)$$
\\\\ \textbf{2-a)}  First, we relate $x_1$ and $x_2$:
$$x_2(t)= x_1(t)-C\sin(\om_d) t$$
Then, we compute the Lagrangian:
$$L=T-V= \f 1 2 (m_1 \dot x_1^2 + m_2 (\dot x_1 - C\om_d \cos(\om_d t))^2)-\f 1 2 (K_1 x_1^2+K_2 (x_1-C \sin(\om_d t)^2))$$
We then compute
$$\pdif L {\dot x_1}= m_1 \dot x_1 + m_2 (\dot x_1 - C \om_d \cos(\om_d t))$$
$$\pdif L {x_1}=-(K_1 x_1+K_2(x_1-C\sin (\om_d t)))$$
The Euler-Lagrange equation is then 
$$  (m_1+m_2)\ddot x_1 + C m_2 \om_d^2 \cos(\om_d t)=-((K_1+K_2) x_1 - C K_2 \sin(\om_d t))$$
\\\\ \textbf{2-b)}  The relation between $x_1$ and $x_2$ we will be using is
$$x_1(t)=x_2(t) + C\sin(\om_d t)$$
Therefore, the Lagrangian for the system is 
$$L=T-V= \f 1 2 (m_2 \dot x_2^2 + m_1 (\dot x_2 + C\om_d \cos(\om_d t))^2)-\f 1 2 (K_2 x_2^2+K_1 (x_2+C \sin(\om_d t)^2))$$
We then compute
$$\pdif L {\dot x_2}= m_2 \dot x_2 + m_1 (\dot x_2+ C \om_d \cos(\om_d t))$$
$$\pdif L {x_2}=-(K_2 x_2+K_1(x_2+C\sin (\om_d t)))$$
The Euler-Lagrange equation is then
$$ (m_1+m_2) \ddot x_2-m_1 C \om_d^2 \sin(\om_d t)=-((K_1+K_2) x_2+C K_1 \sin(\om_d t))$$

\paragraph{Question 3}
\subparagraph{Part 1}
We will assume that Shaun starts standing on the right side by the ramp. Now let us assume that he somehow manages to propel himself forward at some finite speed $v$ to enter the halfpipe (this seems somewhat difficult as he doesn't have arms, there is no friction and in general his motions are rather constrained in our idealizing, but he could perhaps have a friend throw a snowball at him). 
Then his initial gravitational potential energy is equal to
\[ E = mgy = mg(R+l) \]
Now, at the bottom of the halfpipe (again invoking the assumption that $w<<R$, so that the skateboard is perfectly parallel to the bottom of the halfpipe - i.e. Shaun's feet are at at $y=0$), we may use conservation of energy to find Shaun's Potential and Kinetic Energy:
\[ E_{PE} = mg(l-h) \]
\[ E_{KE} = E - E_{PE} = mg[(R-\frac{w}{2})-l+h] \]
Now, by moving his center of mass up by $h$, we have
\[ E'_{PE} = mgl \]
but the kinetic energy remains unchanged. Finally, we may find how high Shaun flies above the half pipe by calculating
\[ \Delta H = \frac{E'_{PE}+E_{KE}-E}{mg} - \frac{w}{2} = h- \frac{w}{2}  \]
\subparagraph{Part 2}
According to our assumptions extending the bottom of the half pipe should not change the theoretical maximum of Shaun flying over the edge of the pipe. However, in our calculation, we assume the he stands up from the crouched position instantaneously. The longer bottom will give him more time to perform this action when it is most efficient (at the bottom of the pipe) and might thus be beneficial to the actual height reached (rather than the theoretical maximum).

\subparagraph{Part 3a)}
Since there is no friction, we know that the only forces acting on Shaun are the normal force and gravity. Now, consider the situation in which Shaun's center of mass is located some distance $z$ below the top of the half pipe. Then we know that
\[ z=R\cos(\theta) \]
\[ F_g = F_n\cos\theta \] 
so the magnitude of $F_n$ is 
\[ F_n = \frac{F_g}{\cos\theta} = \frac{RF_g}{z} \]



\paragraph{Question 3 - 1}
We will start by considering Shaun being positioned exactly horizontally (which we may assume since $w << R$in the upper right corner of the halfpipe.

\paragraph{Question 4.}

\subparagraph{Part 1.}  Moment-of-inertia tensors simply add, so the
total moment-of-inertia tensor is $$\overleftrightarrow{I_\tot}
= \overleftrightarrow{I^o} + \overleftrightarrow{I^i} =
\begin{bmatrix}
I_\bot^i + I_\bot^o & 0 & 0 \\
0 & I_\bot^i + I_\bot^0 & 0 \\
0 & 0 & I_3^i + I_3^o
\end{bmatrix}$$

\subparagraph{Part 2.}  The general solution for a free symmetric top
(Morin's equations 9.50 and 9.51) gives us the following general solution for
the motion of the entire space station: $$\vec{\omega}
= \parens{A \cos (\Omega t + \phi), A \sin (\Omega t
+ \phi), \omega_3}$$ $$\vec{L} = \parens{(I_\bot^i + I_\bot^o) A \cos
(\Omega t + \phi), (I_\bot^i + I_\bot^o) A \sin (\Omega t + \phi),
(I_3^o + I_3^i) \omega_3}$$ where $$\Omega = \parens{\frac{(I_3^o + I_3^i) -
(I_\bot^o + I_\bot^i)}{I} \omega_3}$$ and $A$ is determined by the
magnitude of the off-center component of $\vec{\omega}$.  We can set
$\hat{x}, \hat{y}, \hat{z}$ in our coordinate system such that
$\vec{\omega}$ oscillates about the $z$-axis and has a zero
$\hat{y}$-component at $t=0$, allowing us to set $\phi = 0$.  Let $\theta$
be the constant angle between $\omega_3$ and $\hat{z}$; this implies
$\tan \theta = A (I_\bot^o + I_\bot^i)/(I_3^o + I_3^i)$.

If the exterior and interior of the space station are disconnected,
then they will thereafter behave as two independent free symmetric
tops.  As the ratio of $I_\bot$ to $I_3$ differs for each, they will
no longer necessarily precess around the original axis, and not
necessarily with the original speed.

\end{document}
