\title{Physics 16 Problem Set 1}
\documentclass[12pt,letterpaper]{article}

\usepackage{graphicx}
\usepackage{caption}
\usepackage{subcaption}
\usepackage{amsmath} 
\usepackage{amssymb}
\usepackage{ulem}
\usepackage{tikz}
\usepackage{multicol}
\usepackage{verbatim}
\usepackage[left=1in,top=1in,right=1in,bottom=1in,nohead]{geometry}
\usetikzlibrary{decorations.markings}
\usetikzlibrary{decorations.pathreplacing}

\usepackage{amsthm} 
\usepackage{wrapfig}
\usepackage{enumitem}
%\usepackage{enumerate}
\newtheorem{mydef}{Definition}
\newtheorem{example}{Example}
\newtheorem{thrm}{Theorem}
\newtheorem{lemma}{Lemma}
\newtheorem{cor}{Corollary}
\newtheorem{notation}{Notation}
\newtheorem{rem}{Remarks}
\newcommand{\biu}[1]{\underline{\textbf{\textit{#1}}}}
\newcommand{\so}{\Rightarrow}
\newcommand{\Lagr}{\mathcal{L}}
\usepackage[ampersand]{easylist}

\let\oldemptyset\emptyset
\let\emptyset\varnothing

\author{Keno Fischer}

\newcommand{\homework}{\biu{Homework}}
\newcommand{\Mor}{\text{Mor}}
\newcommand{\N}{\mathbb{N}}
\newcommand{\Q}{\mathbb{Q}}
\newcommand{\Z}{\mathbb{Z}}
\newcommand{\R}{\mathbb{R}}
\newcommand{\C}{\mathbb{C}}
\newcommand{\pabs}[1]{\left|\left| #1 \right|\right|_p}
\newcommand{\set}[1]{\left\{#1 \right\}}
\newcommand{\paren}[1]{\left(#1 \right)}
\newcommand{\parens}{\paren}
\newcommand{\tot}[0]{\text{tot}}
%\newcommand{\pabs}[1]{#1}
\begin{document}
\tikzstyle{lattice}=[shape=circle,draw,fill,text=white]
\tikzset{
  % style to apply some styles to each segment of a path
  on each segment/.style={
    decorate,
    decoration={
      show path construction,
      moveto code={},
      lineto code={
        \path [#1]
        (\tikzinputsegmentfirst) -- (\tikzinputsegmentlast);
      },
      curveto code={
        \path [#1] (\tikzinputsegmentfirst)
        .. controls
        (\tikzinputsegmentsupporta) and (\tikzinputsegmentsupportb)
        ..
        (\tikzinputsegmentlast);
      },
      closepath code={
        \path [#1]
        (\tikzinputsegmentfirst) -- (\tikzinputsegmentlast);
      },
    },
  },
  % style to add an arrow in the middle of a path
  end arrow/.style={postaction={decorate,decoration={
        markings,
        mark=at position 1 with {\arrow[#1]{stealth}}
      }}},
}

\maketitle
\section*{Problem 1-1}
We want to employ dimensional analysis to derive an equation for $v$ as a function of $R$,$\rho$,$S$. To do so it is of utility to write down the dimensions of each of the variables involved:
\begin{eqnarray}
v = \dfrac{1}{time}
R = (length) \\
\rho = \dfrac{(mass)}{(length)^3}
S = \dfrac{(force)}{(length)} = \dfrac{(mass)}{(time)^2}
\end{eqnarray}

We can clearly see that the $v \propto \sqrt{S}$ all that remains is to cancel out mass ($v \propto \sqrt{\frac{S}{\rho}}$) and length ($v \propto \sqrt{\frac{SR^3}{\rho}}$). Finally introducing a constant of proportionality gives us
\[ v=C\sqrt{\dfrac{SR^3}{\rho}} \]
\section*{Problem 1-2}
We will assume a very simple bock stacking in imposing that every block's center of mass needs to lie on the previous block (we're constructing the stack "top-down". Calling the block length $l$, we see that the fist block ($x_cm = \frac{1}{2}l$ may be at most $\frac{1}{2}$ offset from the previous block's center off mass (or left edge from left edge). The x coordinate of the new center for mass of blocks stacked in such a way is $x_{cm_{1+2}}=\frac{\frac{1}{2}l+\frac{3}{2}l}{2}=\frac{1}{4}l$, so the left edge of the second block may be at most $\frac{1}{4}l$ apart from the left edge of the third block. Continuing in such a way, we eventually get that for the $i$ block, the sum of all ofsets is equal to $\sum\limits_{n=1}^i \frac{1}{2n}=\frac{1}{n}\sum\limits_{n=1}^i \frac{1}{n}$ which is half the harmonic series. In particular note that for $i=$, the maximum possible offset is $\frac{1}{2}+\frac{1}{4}+\frac{1}{6}+\frac{1}{8}=\frac{24}{25}$ and the series goes to $\infty$ as $i\to\infty$ (since it's an harmonic series).
\section*{Problem 1-3-a}
The frictional force needs to have dimensions $\frac{(mass)(length)}{(time)^2}$. Note that the velocity has dimensions $\frac{(length)}{(time)}$. Furthermore, we note that the dimensions of $\beta$ must be simply $1$ (since there's no unit of the exponent. 
However, $\alpha$ still needs to cancel the additional exponent to the dimensions of $v$ introduced by $\beta$ and therefore the dimensions of alpha are: 
\[\dfrac{(mass)}{(time)}\left(\dfrac{s}{m}\right)^\beta\]
\section*{Problem 1-3-b}
We first study the case where
\[ F_f = -\alpha \]
In this case the deceleration due to friction will be
\[ a = -\frac{\alpha}{m} \]
, the velocity
\[ v = -\frac{\alpha}{m}t + v_0 \]
and the position
\[ v = -\frac{\alpha}{2m}t^2 + v_0t + x_0 \].
In particular note that unlike the $\beta = 1$ case, the velocity will reach $0$ at $t=\frac{v_0m}{\alpha}$, i.e. in finite time (it still travels a finite amount of space).
\section*{Problem 1-3-c}
The velocity decreases to $0$ in finite time, and thus the object travels a finite distance in this finite amount of time (the reason for this is that for $v'=-v^\beta$, we need not have $v^\beta<v$ if $\beta<1$, so in a certain time $v$ will come to a stop).
\section*{Problem 1-3-d}
Our task is to solve the general problem \[ mv' = -\alpha v^\beta \]
We will first apply separation of variables to obtain
\[ v'v^{-\beta} = -\frac{\alpha}{m} \]
 Now, for $\beta \notin \{0,1\}$, we obtain
 \[ \dfrac{v(t)^{-\beta+1}}{-\beta+1}=-\frac{\alpha}{m}t+C_v \]
 or 
 \[ v(t) = \sqrt[\beta-1]{\frac{1}{(1-\beta)\left(\frac{\alpha}{m}t+C_v\right)}} = \left((1-\beta)\left(\frac{\alpha}{m}t+C_v\right)\right)^{\frac{1}{-\beta+1}} \]
 If we actually want the constant to represent $v(0)$, we can write this as
 \[ v(t) = \left(v_0^{1-\beta}-\dfrac{\alpha(1-\beta)}{m}t\right)^{\frac{1}{1-\beta}} \]
 To summarize this result with the results from above we can say
 \[ v(t) = \begin{cases} -\frac{\alpha}{m}\ t + v_0 &\mbox{if } \beta = 0 \\ 
-v_0e^{-\frac{\alpha}{m} t} & \mbox{if }  \beta = 1 \\
\left(v_0^{1-\beta}-\dfrac{\alpha(1-\beta)}{m}t\right)^{\frac{1}{1-\beta}} & \mbox{if } \beta \notin \{0,1\} \end{cases}  \]
The change in behavior between at $\beta = 1$ from terminating in finite time to to not terminating is easily explained by the change in the sign of the exponential and the coefficient. For $\beta<1$, $v$ will be $0$ at time 
\[ t = \dfrac{v_0^{1-\beta}m}{\alpha(1-b)} \]
(i.e. a finite point in time). For $\beta>1$ the sign of the exponent flips (meaning the inside is inverted), however, since the sign of the coeffiecient of $t$ flips at the same time the denominator of the newly inverted expression will go to $\infty$ rather than $0$ (as it used to for $\beta < 1$), so the entire expression will eventually go to $0$ as $t\to\infty$ (though it will take infinite amounts of time).
\section*{Problem 1-3-e}
Simply integrating the expression we obtained for $v(t)$ in the previous part of the problem, we get
 \[ x(t) = \begin{cases} -\frac{\alpha}{2m}\ t^2 + v_0t + x_0 &\mbox{if } \beta = 0 \\ 
\frac{mv_0}{\alpha}e^{-\frac{\alpha}{m} t} + x_0 & \mbox{if }  \beta = 1 \\
 x_0 + \frac{m v_0^{2-\beta}}{\alpha(2-\beta)} + \dfrac{\left(-v_0^{1-\beta}+m\dfrac{\alpha(1-\beta)}{m}t\right)}{\alpha(2-\beta)}^{\frac{2-\beta}{1-\beta}} & \mbox{if } \beta \notin \{0,1,2\} \end{cases}  \]
 We already know that $x(t)$ must be finite in the limit for $\beta\leq 1$, since $v(t)$ hits $0$ in finite time furthermore note that for $1 < \beta$, the integral is of the form $\int \frac{1}{x^(\beta-1)}$ and by the p integral test it converges for  $1<\beta<2$ diverges for $2\leq \beta$. To find the distance traveled for the first case, we can take the time at which $v=0$ (which is $t=\frac{v_0m}{\alpha}$; see above), so the distance traveled for $\beta=0$ is:
 \[ x_{\beta=0} = \frac{v_0^2m}{2\alpha} + x_0 \]
 We can do the same thing for $0<\beta<1$ (as it terminates in finite time), so for $t= \dfrac{v_0^{1-\beta}m}{\alpha(1-\beta)}$ we get
 \[ x_{0<\beta<1} = x_0 + \dfrac{mv_0^{2-\beta}}{\alpha(2-\beta)}\]
For $\beta = 1$, we need to take the limit as $t\to\infty$ which gives us:
\[\lim\limits_{t\to\infty} x_{\beta=1} =\frac{mv_0}{\alpha} \]
Finally for $1<\beta<2$, we  also get
\[ \lim\limits_{t\to\infty} x_{1<\beta<2} = x_0 + \dfrac{mv_0^{2-\beta}}{\alpha(2-\beta)} \]
as the denominator under the exponent goes to $-\infty$ (and thus the whole expression goes to $0$.
\section*{Problem 1-3-f}
We want to integrate $dx=\dfrac{v dv}{a}=\frac{mv^{1-\beta}}{\alpha}$, gives us 
\[ x = \frac{mv^{2-\beta}}{\alpha(2-\beta)}\]
for v which go to $0$.
As expected, this evaluates to 
 \[ x_{\beta=0} = \frac{v_0^2m}{2\alpha} \]
 \[ x_{\beta=1} = \frac{mv_0}{\alpha} \]
 \[ x_{1<\beta<2} = x_{0<\beta<1} =  \frac{mv^{2-\beta}}{\alpha(2-\beta)} \]
 There are no problems at $\beta = 1$. In this case we know that $v$ will always approach $0$ rather than having to deal with two different cases (one in which $t$ is finite and one in which it is inifinite - which happens precisely at the $\beta=1$ boundary.
\end{document}